Die Konstruktion der Hardware ist in den folgenden Kapitlen auf drei Überkategorien 
an Arbeitsschritten aufgeteilt, welche in chronologischer Reihenfolge auszuführen sind. 
Innerhalb der Kategorien ist der Ablauf nicht notwendigerweise streng einzuhalten, 
allerdings ist empfohlen, logisch zu denken, bevor gehandelt wird.

\subsection{Druck der Komponenten}
\label{ssec:print}
Bevor mit dem Aufbau begonnen werden kann, müssen die entsprechenden Komponenten 3D-gedruckt 
werden. Dazu wird ein SLA-Drucker empfohlen, aus Gründen, die in Kapitel \ref{ssec:Print} genannt wurden. 
Da keine Experimente mit FDM-Druckern durchgeführt wurden, ist nicht garantiert, dass damit nutzbare 
Komponenten produziert werden können.
Bei Details zu den Modellen wird auf Kapitel \ref{ssec:Models} verwiesen.
\\
Die STL-Dateien stehen auf dem folgend verlinkten GitHub zur Verfügung:

\url{https://github.com/psykovski-extended/binobo_models.git}
\\\\
Da es viele verschiedene Drucker und damit verbundene Software gibt, sind diese Files selbst zu 
slicen. Tabelle \ref{tab:comp} gibt unbedingt notwendige Komponenten und deren Stückzahl an:

\begin{table}[H]
    \centering
    \begin{tabular}{|l|r|}
        \hline
        Name&Stückzahl\\
        \hline
        \hline
        controller$\_$poti$\_$holder&6\\
        \hline
        controller$\_$poti$\_$holder$\_$th&1\\
        \hline
        controller$\_$poti$\_$holder$\_$p&1\\
        \hline
        controller$\_$poti$\_$holder$\_$p$\_$cut&1\\
        \hline
        controller$\_$poti$\_$fingertip&3\\
        \hline
        controller$\_$poti$\_$fingertip$\_$th&1\\
        \hline
        controller$\_$poti$\_$fingertip$\_$p&1\\
        \hline
        controller$\_$poti$\_$thumb$\_$base&1\\
        \hline
        controller$\_$plate$\_$inner&1\\
        \hline
        controller$\_$plate$\_$thumb&1\\
        \hline
        controller$\_$plate$\_$outer&1\\
        \hline
        controller$\_$strut$\_$lr M&4\\
        \hline
        controller$\_$strut$\_$lr S&1\\
        \hline
    \end{tabular}
    \caption{Liste erforderlicher Komponenten}
    \label{tab:comp}
\end{table}

Die Größe der benötigten Hebel und Streben ist abhängig von der ungefähren Größe der Hand. Weil 
jeder Mensch anders ist, kann nicht mit 100\% Genauigkeit angegeben werden, welche dieser Teile 
für die entsprechende Hand passen. Daher sind hier nur die Maße des Prototypen angegeben. Bei 
der Montage ist es möglicherweise erforderlich weitere Streben und Hebel nachzudrucken, wenn andere 
Maße notwendig sind.

\begin{table}[H]
    \centering
    \begin{tabular}{|l|r|}
        \hline
        Name&Stückzahl\\
        \hline
        \hline
        controller$\_$strut L&1\\
        \hline
        controller$\_$strut M&2\\
        \hline
        controller$\_$strut S&7\\
        \hline
        controller$\_$strut XS&5\\
        \hline
        \hline
        controller$\_$lever L&5\\
        \hline
        controller$\_$lever M&9\\
        \hline
        controller$\_$lever S&1\\
        \hline
    \end{tabular}
    \caption{Liste varabel erforderlicher Komponenten}
    \label{tab:}
\end{table}

Die Hebel (\texttt{controller$\_$lever}) sind als Paare angegeben, da in den entsprechenden 
Dateien auch beide Komponenten inkludiert sind.
\\\\
Anzumerken ist, dass die Teile für eine geringe Druckzeit optimiert sind und daher größtenteils 
ohne Stützkonstruktion zu drucken sind. Dabei ist die Orientierung aus den STL-Files zu übernehmen.
Die Halterungen am Fingerrücken stellen eine Ausnahme dar, da diese eine Lasche zur Befestigung des nächsten 
Moduls aufweisen, welche nicht ohne Stützen erfolgreich gedrückt werden kann.

Somit müssen bei diesen Elementen folgende oder vergleichbare Stützstrukturen hinzugefügt werden:

\begin{minipage}{0.5\textwidth}
    \begin{figure}[H]
        \centering
        \includegraphics [angle=0, width=0.8\textwidth]{Images/supports}
        \caption{Konfiguration der Stützen bei controller$\_$poti$\_$holder}
        \label{fig:supports}
    \end{figure}
\end{minipage}
\begin{minipage}{0.5\textwidth}
In Abbildung \ref{fig:supports} wird das empfohlene Stützmuster anhand der Komponente controller$\_$poti$\_$holder 
dargestellt. Bei allen ähnlichen Modellen mit diesen Laschen müssen Stützen nach diesem Vorbild nachgerüstet werden.

Je nach Druckereigenschaften ist es notwendig den Überhang bei den Fingerspitzen ebenfalls zu stützen.
\end{minipage}

Sobald die Komponenten ausgedruckt sind, kann mit deren Montage begonnen werden.

\subsection{Montage der Komponenten}
\label{ssec:mounting}
Es wird empfohlen, die Konstruktion zu zweit durchzuführen, da Nähen in Nähe der eigenen primären Hand alleine 
kompliziert ist.

Die Montage der Potentiometer verläuft leichter, wenn sie vor der Befestigung am Handschuh erfolgt, daher ist empfohlen, damit zu beginnen.
Am einfachsten verläuft dies, wenn man dieser Bewegung folgt:

\begin{minipage}{0.5\textwidth}
\begin{figure}[H]
    \centering
    \includegraphics [angle=0, width=0.8\textwidth]{Images/poti_1.png}
    \caption{Montage: Schritt 1}
    \label{fig:step1}
\end{figure}
\end{minipage}
\begin{minipage}{0.5\textwidth}
    \begin{figure}[H]
        \centering
        \includegraphics [angle=0, width=0.8\textwidth]{Images/poti_2.png}
        \caption{Montage: Schritt 2}
        \label{fig:step2}
    \end{figure}
\end{minipage}

Bei der Montage ist etwas Kraft von Nöten, da die Potentiometer nur durch die unterdimensionierte Lücke in der Halterung fixiert werden. 
Es ist auch darauf zu achten, die Potentiometer richtig herum zu implementieren, sodass sie in die Aussparungen an der Strebe passen. 
Auf der Daumenbasisplatte ist das Potentiometer ebenfalls nach diesem System anzustecken.
Beim unteren Daumengelenk, \texttt{controller$\_$poti$\_$thumb$\_$base}, muss der Widerstand seitwärts auf eine ähnliche weise in die Passform gepresst werden.
\\\\
Die Potentiometer, die an der Basis der normalen Finger liegen müssen erst mit den entsprechenden \texttt{controller$\_$strut$\_$lr}-Varianten 
beaufschlagt werden, bevor sie in ihre Position geschoben werden, da erst durch diese Komponente die erforderliche Höhe erreicht wird, 
um den Teil zu stabilisieren.

Am unteren Daumengelenk muss ebenfalls ein \texttt{controller$\_$strut$\_$lr M}-Element aufgesteckt werden.
\\\\
Nun wird es Zeit für den ersten komplexeren Arbeitsschritt: Das Annähen der Elemente.
Es wird empfohlen, einen flexiblen, engen Handschuh zu nutzen, da es ansonsten zum Abheben der Komponenten von der Hand kommen kann.
\\
Die Platzierung der Komponenten ist zum Teil nutzerabhängig, allerdings ist die Reihenfolge fix vorgegeben. 
Diese ist in folgender Grafik, Abbildung \ref{fig:scheme}, gegeben:

\begin{figure}[H]
    \centering
    \includegraphics [angle=0, width=0.8\textwidth]{Images/scheme.png}
    \caption{Anordnungsschema auf dem Handschuh}
    \label{fig:scheme}
\end{figure}

Alle Halterungen sollten möglichst mittig auf den unterliegenden Fingerknochen liegen und lotrecht nach oben stehen. Empfohlen ist, 
die Halterungen an den Fingern zuerst zu befestigen, bevor die Platten und \texttt{controller$\_$poti$\_$thumb$\_$base} montiert werden, 
da deren genaue Position und Orientierung von ihnen abhängig sind. 

Nachdem die Teile angenäht und angeschraubt wurden, müssen die beiden großen Platten mit den Multiplexer-Boards beaufschlagt werden. dazu werden sie, 
nach einer Mutter, die als Abstandhalter dient, mit den Löchern auf die beiden handinneren Schrauben gesteckt und mit einer weiteren Mutter befestigt.
\\\\
Als letzter Schritt dieser Etappe müssen passende Hebel- und Strebenlängen gewählt werden. Man erkennt die korrekte Längenkombination daran, dass beim 
Abbiegen des Fingers kaum spürbarer Widerstand beim Erreichen des vollen Winkels spürbar ist und die am Potentiometer verfügbaren 90° voll ausgekostet 
werden. Hierbei kann es erforderlich sein weitere Komponenten zu drucken. Auch wichtig anzumerken ist, dass die dickeren Bottom-Layers, die beim SLA-Druck 
auftreten an Löchern manchmal aufgestochen oder aufgeritzt werden müssen. Weiters ist wichtig, dass die Hebel, besonders die, die in das Potentiometer 
greifen, vollständig ausgehärtet sind, bevor sie verbaut werden, da es andernfalls zu Abrieb und Verschleiß kommen kann. 


\subsection{Verlöten der Elektronik}
\label{ssec:soldering}

Zuletzt müssen die Verbindungen zwischen den Widertständen und den Multiplexern gelötet werden. Dazu werden hochflexible Drähte empfohlen, beispielsweise 
\textit{Flexivolt 0.25mm}-Drähte, wie sie im Prototypen genutzt werden. Tabelle \ref{tab:circuit} gibt an, wie die Läuferkontakte der Potentiometer in jedem Segment zu verschalten sind:

\begin{table}[H]
    \centering
    \begin{tabular}{|l|l|}
        \hline
        Segment&Kontakt\\
        \hline
        \hline
        &Multiplexer 1\\
        \hline
        Daumen auf Platte&S0\\
        \hline
        Daumen Links/Rechts&S1\\
        \hline
        Daumen Mitte&S2\\
        \hline
        Daumenspitze&S3\\
        \hline
        Zeigefinger Links/Rechts&S4\\
        \hline
        Zeigefinger unten&S5\\
        \hline
        Zeigefinger Mitte&S6\\
        \hline
        Zeigefingerspitze&S7\\
        \hline
        Mittelfinger Links/Rechts&S8\\
        \hline
        Mittelfinger unten&S9\\
        \hline
        Mittelfinger MItte&S10\\
        \hline
        Mittelfingerspitze&S11\\
        \hline
        \hline
        &Multiplexer 2\\
        \hline
        Ringfinger Links/Rechts&S0\\
        \hline
        Ringfinger unten&S1\\
        \hline
        Ringfinger Mitte&S2\\
        \hline
        Ringfingerspitze&S3\\
        \hline
        Kleiner Finger Links/Rechts&S4\\
        \hline
        Kleiner Finger unten&S5\\
        \hline
        Kleiner Finger Mitte&S6\\
        \hline
        Kleiner Finger Spitze&S7\\
        \hline
    \end{tabular}
    \caption{Verschaltungstabelle der Läuferkontakte}
    \label{tab:circuit}
\end{table}

Die anderen Kontakte der Potentiometer müssen mit 5V bzw GND beaufschlagt werden, sodass ein Spannungsteiler 
entsteht, um die Rotation zu messen. Obwohl die Polung technisch irrelevant ist, 
müssen die Widerstände richtig gepolt sein, um Vorzeichenfehler und falsche Messungen zu vermeiden. Das Programm 
gibt vor, dass immer der obere Kontakt des Potentiometers auf 5V angeschlossen ist, der übrige auf GND.
Bei den liegenden Potentiometern wird das selbe, nur gedreht angenommen, also links GND und rechts 5V. 
Da an dieser Stelle zweimal je 25 Drähte miteinander verbunden werden müssen, ist eine Form von Spannungsschiene 
stark zu empfehlen. Beim Prototypen wurden zwei Stücke einer querverbundenen Lochrasterplatte genutzt, auf der 
alle Verbindungen der entsprechenden Spannung zusammengeführt wurden. Die Anschlüsse für den ESP32 dürfen nicht vergessen werden. 
Beim Prototypen wurden Jumper-Wires genutzt, um den $\mu$C schnell austauschen zu können, doch bei einem permanenten 
Projekt können diese Verbindungen auch gelötet werden.
\\\\
Auch die Multiplexer benötigten Spannung, daher müssen sie ebenfalls an der Spannungsschiene angeschlossen werden.
der EN-Pin ist auf GND zu ziehen, da dieser einen negierten Enable-Pin darstellt, somit ist der Chip permanent aktiv, wenn 
auf EN 0V anliegen.
\\\\
Die restlichen Pins des Multiplexers müssen wie in Tabelle \ref{tab:multiESP} folgt mit dem ESP32 verbunden werden:

% Table generated by Excel2LaTeX from sheet 'Tabelle1'
\begin{table}[htbp]
  \centering
  
    \begin{tabular}{|l|r|r|}
    \hline
          & \multicolumn{2}{c|}{ESP32 Pin} \\
    Pin Name & \multicolumn{1}{l|}{Multiplexer 1} & \multicolumn{1}{l|}{Multiplexer 2} \\
    \hline
    S0    & 25    & 23 \\
    \hline
    S1    & 33    & 22 \\
    \hline
    S2    & 32    & 21 \\
    \hline
    S3    & 12    & 19 \\
    \hline
    Z     & 34    & 35 \\
    \hline
    EN    & 26    & 5 \\
    \hline
    \end{tabular}%
    \caption{Löteinteilung der Multiplexer}
  \label{tab:multiESP}%
\end{table}%

Bevor der Controller in Betrieb genommen wird, sollte mit einem Ohmmeter der Widerstand zwischen den 5V und GND-Leitungen gemessen werden. 
Er sollte sich mindestens im zweistelligen Kiloohm-Bereich befinden, ansonsten fällt der ESP32 in einen Reset-Loop durch den Kurzschluss 
an der Versorgungsspannung.
\\\\
Somit muss der $\mu$C nur noch mit der entsprechenden Software beaufschlagt werden und mit einem Handy und der Binobo-App verbunden werden, 
um den Controller in Betrieb zu nehmen.
\pagebreak
Zuletzt folgen noch zwei Bilder des vollständigen Controllers, Abbildung \ref{fig:mt} und \ref{fig:gauntlet}.
\begin{figure}[H]
    \centering
    \includegraphics [angle=0, width=0.59\textwidth]{Images/Hand_leer.png}
    \caption{Leerer, vollständiger Handschuh}
    \label{fig:mt}
\end{figure}

\begin{figure}[H]
    \centering
    \includegraphics [angle=0, width=0.59\textwidth]{Images/Hand.png}
    \caption{Angezogener Handschuh}
    \label{fig:gauntlet}
\end{figure}
