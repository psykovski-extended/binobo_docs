\selectlanguage{naustrian}

\begin{figure}[H]
    \begin{subfigure}[b]{0.5\textwidth}
            \includegraphics[width=\textwidth]{Images/teile/controller_poti_holder/side.png}
            \centering
        
    \end{subfigure}
    \begin{subfigure}[b]{0.5\textwidth}
            \includegraphics[width=\textwidth]{Images/teile/controller_poti_holder/front.png}
            \centering
    \end{subfigure}
    
    \begin{subfigure}[b]{0.6\textwidth}
        \begin{table}[H]
            \centering
            \begin{tabular}{|c|c|}
                \hline
                Name&$\texttt{controller}\_ \texttt{poti}\_ \texttt{holder}$\\
                \hline
                LxBxH&19.9mm x 24.2mm x 28.3mm\\
                \hline
                \vspace{3mm}
                Beschreibung&\shortstack{Dieses Element erfüllt die Rolle,\\ein Potentiometer auf den\\Fingerrücken zu fixieren.}\\
                &\shortstack{Das Modul ist für die meisten\\Finger verwendbar, da sie eine\\ähnliche Größe aufweisen-\\Die Aussparung in der Mitte\\dient als Kanal für die Drähte,\\welche die Gelenke mit der\\Messelektronik verbinden}\\
                \hline
            \end{tabular} 
            \caption{Details:controller-poti-holder}
            
        \end{table}s
        
    \end{subfigure}
    \begin{minipage}{0.4\textwidth}
        Da die Finger der menschlichen Hand verschiedene Breiten aufweisen, ist es erforderlich angepasste Elemente für 
        Daumen und Kleinen Finger zu drucken. Diese sind dem Grundmodul ähnlich, doch der Winkel und die Breite der Unterseite sind an die entsprechenden Finger angepasst. 
    \end{minipage}
\end{figure}



\begin{minipage}{0.5\textwidth}
    \begin{figure}[H]
        \includegraphics[width=\textwidth]{Images/teile/controller_poti_holder/p_front.png}
        \centering
        \caption{Version: Kleiner Finger}
    \end{figure}
\end{minipage}
\begin{minipage}{0.5\textwidth}
    \begin{figure}[H]
        \includegraphics[width=\textwidth]{Images/teile/controller_poti_holder/th_front.png}
        \centering
        \caption{Version: Daumen}
    \end{figure}
\end{minipage}

Aufgrund des Aufbaus des Handschuhes ist es erforderlich, bei der untersten Haltevorrichtung des Kleinen Fingers einen Ausschnitt an der 
Innenseite einzuplanen, da ansonsten die Bewegungsfreiheit und der Tragekomfort reduziert wird.

\begin{minipage}{0.5\textwidth}
    \begin{figure}[H]
        \includegraphics[width=\textwidth]{Images/teile/controller_poti_holder/p_top}
        \centering
        \caption{Version: Ausschnitt}
        \label{cut}
    \end{figure}
\end{minipage}
\begin{minipage}{0.5\textwidth}
    In Abbildung \ref{cut} ist der Ausschnitt zu sehen, der dazu dient, Reibung und Verhaken von 
    Handschuh und Konstruktion zu vermeiden. Die durch das fehlende Loch verlorene Stabilität 
    ist hierbei in Kauf zu nehmen. 
    
\end{minipage}

Um zusätzliche Stabilität an den Fingerspitzen zu gewährleisten, wird bei den entsprechenden Modulen eine Art Kappe vorgesehen, die die Fingerspitze abdeckt. 
Weiters ist an diesen Elementen keine Aufhängung für weitere Gelenke notwendig. 
\begin{minipage}{0.5\textwidth}
    \begin{figure}[H]
        \includegraphics[width=\textwidth]{Images/teile/controller_poti_holder/tip_side.png}
        \centering
        \caption{Seitenansicht}
    \end{figure}
\end{minipage}
\begin{minipage}{0.5\textwidth}
    \begin{figure}[H]
        \includegraphics[width=\textwidth]{Images/teile/controller_poti_holder/tip_front.png}
        \centering
        \caption{Frontansicht}
    \end{figure}
\end{minipage}
\begin{minipage}{0.6\textwidth}
    \begin{table}[H]
        \centering
        \begin{tabular}{|c|c|}
            \hline
            Name&$\texttt{controller}\_ \texttt{poti}\_\texttt{fingertip}$\\
            \hline
            LxBxH&21.8mm x 23.9mm x 24.4mm\\
            \hline
            Beschreibung&\shortstack{Dieses Element erfüllt die Rolle, \\ein Potentiometer auf der \\Fingerspitze zu fixieren.}\\
            &\shortstack{Die Abdeckung an der Spitze\\ dient der Stabilität gegen\\längsseitiges Kippen und\\erhöht die Auflagefläche}\\
            \hline
        \end{tabular} 
        \caption{Details:controller-poti-fingertip}
        \label{tab:test}
    \end{table}
    
\end{minipage}
\begin{minipage}{0.4\textwidth}
    Dieses Element erfüllt auch einen weiteren Zweck, da beim Annähen der Spitze überschüssiger Stoff des Handschuhes 
    gebunden werden kann, um die Straffheit zu gewährleisten, die der Stabilisierung der hinteren Fingerelemente dient, da 
    dadurch verhindert wird, dass sie sich von der Haut aufgrund der Flexibilität des Stoffes abheben und somit 
\end{minipage}


Messergebnisse verfälschen. Selbstverständlich sind auch hier alternative Modelle für die äußersten Finger erforderlich. Hier wurde nicht nur die Breite der Auflagefläche modifiziert, sondern auch die Kappe an der Spitze.

\begin{minipage}{0.5\textwidth}
    \begin{figure}[H]
        \includegraphics[width=\textwidth]{Images/teile/controller_poti_holder/tip_p.png}
        \centering
        \caption{Version: Kleiner Finger}
    \end{figure}
\end{minipage}
\begin{minipage}{0.5\textwidth}
    \begin{figure}[H]
        \includegraphics[width=\textwidth]{Images/teile/controller_poti_holder/tip_th.png}
        \centering
        \caption{Version: Daumen}
    \end{figure}
\end{minipage}


Für den Daumen war aufgrund der Komplexität des unteren Gelenks eine Spezialanfertigung notwendig. 
Dieses Halterung muss den Hebel zur Platte am Handrücken halten, doch gleichzeitig auch in einer anderen Achse ein Potentiometer fixieren, um die Greifbewegung des Daumens korrekt aufzunehmen. 
Diese Herausforderung wurde bewältigt, indem das folgende Modul durch iteratives Testen entwickelt wurde:
\begin{minipage}{0.5\textwidth}
    \begin{figure}[H]
        \includegraphics[width=\textwidth]{Images/teile/controller_poti_holder/th_base_side.png}
        \centering
        \caption{Seitenansicht}
    \end{figure}
\end{minipage}
\begin{minipage}{0.5\textwidth}
    \begin{figure}[H]
        \includegraphics[width=\textwidth]{Images/teile/controller_poti_holder/th_base_top}
        \centering
        \caption{Oberansicht}
    \end{figure}
\end{minipage}
\begin{minipage}{0.6\textwidth}
    \begin{table}[H]
        \centering
        \begin{tabular}{|c|c|}
            \hline
            Name&$\texttt{controller}\_ \texttt{poti}\_\texttt{thumb}\_\texttt{base}$\\
            \hline
            LxBxH&14.9mm x 35.1mm x 12.5mm\\
            \hline
            Beschreibung&\shortstack{Dieses Element erfüllt die Rolle, \\als Übergang von der Platte\\zu den Daumengelenken zu}\\
            &\shortstack{agieren. Es ist über ein\\direktes Gelenk, sowie über\\einen Hebel mit der Basis-\\Platte verbunden.}\\
            \hline
        \end{tabular} 
        \caption{Details:controller-poti-fingertip}
        \label{tab:th_base}
    \end{table}
\end{minipage}