Bei der Integrierung der Potentiometer werden Hebel genutzt, die genau in den sechseckigen Ausschnitt des 
Schleifers passen (Abbildung \ref{levertop} bis \ref{leverprint}). Diese müssen sehr genau bemessen und gedruckt sein, da es sonst zu Abrieb des Epoxidharzes, 
oder zu Komplikationen bei der Konstruktion kommen kann. Um Kompatibilität für die verschiedenen Gelenke und 
Gelenkabstände sicherzustellen gibt es drei Ausführungen dieser Hebel in verschiedenen Längen. Zu jeder Länge 
wird auch ein Gegenstück bereitgestellt, das für verbesserte Stabilität sorgt.

\begin{minipage}{0.5\textwidth}
    \begin{figure}[H]
        \centering
        \includegraphics [angle=0, width=0.8\textwidth]{Images/teile/levers_struts/levers_top.png}
        \caption{Obersicht Hebel}
        \label{levertop}
    \end{figure}
    \end{minipage}
    \begin{minipage}{0.5\textwidth}
        \begin{figure}[H]
            \centering
            \includegraphics [angle=0, width=0.8\textwidth]{Images/teile/levers_struts/levers_front.png}
            \caption{Frontansicht Hebel}
            \label{leverfront}
        \end{figure}    
    \end{minipage}

\begin{minipage}{0.5\textwidth}
    \begin{table}[H]
        \centering
        \begin{tabular}{|c|c|}
            \hline
            Name&$\texttt{controller}\_ \texttt{lever}$\\
            \hline
            LxBxH L&31.5mm x 13.8mm x 5.2mm\\
            \hline
            LxBxH M&26.5mm x 13.8mm x 5.2mm\\
            \hline
            LxBxH S&17.9mm x 13.8mm x 5.2mm\\
            \hline
        \end{tabular} 
        \caption{Details Hebel}
    \end{table}
\end{minipage}
\begin{minipage}{0.5\textwidth}
    \begin{figure}[H]
        \centering
        \includegraphics [angle=0, width=0.8\textwidth]{Images/teile/levers_struts/levers_r.png}
        \caption{ausgedruckte Hebel}
        \label{leverprint}
    \end{figure}
\end{minipage}

\pagebreak

In Kombination mit diesen Hebeln werden Streben (Abbildung \ref{struts} und \ref{strutsprint}) genutzt, die dazu dienen, beim Abbiegen des zu messenden Gelenks 
dem Finger auszuweichen, ohne die Messung zu behindern. Die Potentiometer werden nicht an der Seite der Finger angebracht, 
da dies die Bewegungsfreiheit der Finger einschränken, und somit den Tragekomfort deutlich reduzieren würde. 
Die Streben wurden in 5 Größen kreiert, um für alle Gelenke bei den meisten Handgrößen anwendbar zu sein. 
Bei der kleinsten Variante ist zusätzlich der Winkel anders, um genauere Messungen zu erzielen. 

\begin{minipage}{0.5\textwidth}
\begin{figure}[H]
    \centering
    \includegraphics [angle=0, width=0.8\textwidth]{Images/teile/levers_struts/struts.png}
    \caption{Obersicht Streben}
    \label{struts}
\end{figure}
\end{minipage}
\begin{minipage}{0.5\textwidth}
    \begin{figure}[H]
        \centering
        \includegraphics [angle=0, width=0.8\textwidth]{Images/teile/levers_struts/struts_r.png}
        \caption{ausgedruckte Streben}
        \label{strutsprint}
    \end{figure}
\end{minipage}

\begin{minipage}{0.6\textwidth}
    \begin{table}[H]
        \centering
        \begin{tabular}{|c|c|}
            \hline
            Name&ausgedruckte Streben\\
            Datei&$\texttt{controller}\_ \texttt{strut}$\\
            \hline
            LxBxH XL&31.1mm x15.3mm x 3.8mm\\
            \hline
            LxBxH L&30.3mm x 14.7mm x 3.8mm\\
            \hline
            LxBxH M&27.6mm x 12.8mm x 3.8mm\\
            \hline
            LxBxH S&24.4mm x 11.4mm x 3.8mm\\
            \hline
            LxBxH XS&19.8mm x 11.3mm x 3.8mm\\
            \hline
        \end{tabular} 
        \caption{Details Streben}
    \end{table}
\end{minipage}
\begin{minipage}{0.4\textwidth}
    An allen Streben und Hebeln sind Löcher mit einem Durchmesser von 2.2mm vorgesehen, welche
    für M2-Schrauben geeignet sind. Obwohl Tests bewiesen haben, dass print-in-place-Gelenke 
    möglich wären, würde dies gegen die Modulare Struktur sprechen, die es ermöglicht, auf 
    vielen Verschie-
\end{minipage}
denen Händen nutzbar zu sein. Somit werden Schrauben benötigt, um die Teile zu verbinden.

\pagebreak

Um die Rotation der Finger aufzunehmen werden die untersten Streben nicht direkt mit der 
Basisplatte verbunden, sondern über ein weiteres Potentiometer. Diese Verbindung erfolgt über 
das folgende Modul auf Abbildung \ref{fbaseside} bis \ref{fbaseprint} dargestellt ist:

\begin{minipage}{0.5\textwidth}
    \begin{figure}[H]
        \centering
        \includegraphics [angle=0, width=0.8\textwidth]{Images/teile/levers_struts/lrhinge_side.png}
        \caption{Seitenansicht Fingerbasisgelenk}
        \label{fbaseside}
    \end{figure}
    \end{minipage}
    \begin{minipage}{0.5\textwidth}
        \begin{figure}[H]
            \centering
            \includegraphics [angle=0, width=0.8\textwidth]{Images/teile/levers_struts/lrhinge_top.png}
            \caption{Obersicht Fingerbasisgelenk}
            \label{fbasetop}
        \end{figure}    
    \end{minipage}

\begin{minipage}{0.5\textwidth}
    \begin{table}[H]
        \centering
        \begin{tabular}{|c|c|}
            \hline
            Name&Fingerbasisgelenk\\
            \hline
            Datei&$\texttt{controller}\_ \texttt{strut}\_\texttt{lr}$\\
            \hline
            LxBxH M&28.4mm x 11.9mm x 7.46mm\\
            \hline
            LxBxH S&26mm x 11.9mm x 7.46mm\\
            \hline
        \end{tabular}
        \caption{Details Fingerbasisgelenke}
    \end{table}
\end{minipage}
\begin{minipage}{0.5\textwidth}
    \begin{figure}[H]
        \centering
        \includegraphics [angle=0, width=0.8\textwidth]{Images/teile/levers_struts/lrhinge.png}
        \caption{ausgedruckte Fingerbasisgelenke}
        \label{fbaseprint}
    \end{figure}
\end{minipage}
