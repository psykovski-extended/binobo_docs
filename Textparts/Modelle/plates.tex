Um einen Bezugspunkt für die Gelenke zu haben und eine Montagemöglichkeit
für die Multiplexer-Boards zur Verfügung zu stellen werden drei Platten am Handrücken befestigt.
Aufgrund einer Kombination aus Praktikabilität beim 3D-Druck und Tragekomfort,
kann hierfür nicht eine einzelne, große Fläche genutzt werden. Das innere Modul ist auf den folgenden Abbildungen \ref{innerplateside} bis \ref{innerplatetop} dargestellt

\begin{minipage}{0.5\textwidth}
    \begin{figure}[H]
        \centering
        \includegraphics [angle=0, width=0.8\textwidth]{Images/teile/plates/plate_inner_side.png}
        \caption{Seitenansicht innere Platte}
        \label{innerplateside}
    \end{figure}
\end{minipage}
\begin{minipage}{0.5\textwidth}
    \begin{figure}[H]
        \centering
        \includegraphics [angle=0, width=0.8\textwidth]{Images/teile/plates/plate_inner_front.png}
        \caption{Frontansicht innere Platte}
    \end{figure}
\end{minipage}

\begin{minipage}{0.4\textwidth}
    \begin{figure}[H]
        \centering
        \includegraphics [angle=0, width=\textwidth]{Images/teile/plates/plate_inner_top.png}
        \caption{Obersicht innere Platte}
        \label{innerplatetop}
    \end{figure}
\end{minipage}
\begin{minipage}{0.6\textwidth}
    \begin{table}[H]
        \centering
        \begin{tabular}{|c|c|}
            \hline
            Name         & innere Handrückenplatte                                 \\
            \hline
            Datei        & $\texttt{controller}\_ \texttt{plate}\_ \texttt{inner}$ \\
            \hline
            LxBxH        & 60.1mm x 43.9mm x 12.6mm                                \\
            \hline
            Beschr.      & \shortstack{Diese Platte erfüllt die Rolle,\\als Basis für  Messungen, sowie\\einen Montagepunkt für die Mul-}\\
                         & \shortstack{tiplexer zu bieten. Das Gelenk\\auf der Innenseite dient als\\Befestigung für das in Tabelle \ref{tab:th_base}\\beschriebene Daumengelenk.} \\
            \hline
        \end{tabular}
        \caption{Details innere Platte}
    \end{table}
\end{minipage}

\vspace{3mm}

Um die Hautverformung bei der Beugung des Daumengelenks nicht in die Messung einfließen zu
lassen, ist die Platte zweigeteilt, sodass nur der relevante Teil des Winkels aufgenommen wird. Der Teil, auf dem das Daumengelenk sitzt wird in Abbildung \ref{thplatefront} und \ref{thplatetop} dargestellt.
Hierbei ist es wichtig, dass die beiden Platten fest am Handschuh befestigt sind, da kleine Bewegungen der
Platten große Messfehler verursachen können. Das vollständige Gelenk ist in Abbildung \ref{thcoup} abgebildet.

\begin{minipage}{0.5\textwidth}
    \begin{figure}[H]
        \centering
        \includegraphics [angle=0, width=0.8\textwidth]{Images/teile/plates/th_base_plate_front.png}
        \caption{Frontansicht Daumenplatte}
        \label{thplatefront}
    \end{figure}
\end{minipage}
\begin{minipage}{0.5\textwidth}
    \begin{figure}[H]
        \centering
        \includegraphics [angle=0, width=0.8\textwidth]{Images/teile/plates/th_base_plate_top.png}
        \caption{Obersicht Daumenplatte}
        \label{thplatetop}
    \end{figure}
\end{minipage}

\begin{minipage}{0.4\textwidth}
    \begin{figure}[H]
        \centering
        \includegraphics [angle=0, width=0.8\textwidth]{Images/teile/plates/thumb_coupling.png}
        \caption{Daumenkoppelung}
        \label{thcoup}
    \end{figure}
\end{minipage}
\begin{minipage}{0.6\textwidth}
    \begin{table}[H]
        \centering
        \begin{tabular}{|c|c|}
            \hline
            Name&Daumenplatte\\
            \hline
            Datei& $\texttt{controller}\_ \texttt{plate}\_ \texttt{thumb}$ \\
            \hline
            LxBxH        & 3.89mm x 29mm x 14.9mm                                  \\
            \hline
            Beschr. & \shortstack{Dieses Element erfüllt die Rolle,\\Daumenverformungen abzufangen,\\die den gemessenen Wert verfäl-}\\
                         & \shortstack{schen könnten. Das festere\\Gelenk verhindert dabei eine\\Drehung, die die Messung am\\nächsten Element beeinflussen\\würde.} \\
            \hline
        \end{tabular}
        \caption{Details Daumenplatte}
    \end{table}
\end{minipage}

\vspace{3mm}

Auf der äußeren Seite der Hand ist eine weitere Platte befestigt (Abbildung \ref{outerfront} bis \ref{outerprint}), welche dazu dient,
einen Basispunkt für Ringfinger und kleinen Finger zu bieten. Dies kann nicht über die
innere Platte erfolgen, da sich die menschliche Hand im normalen Gebrauch so verformt, dass eine geschlossene
Fläche am Handrücken die Bewegungsfreiheit einschränken würde. Somit werden zwei
unabhängig bewegliche Teile genutzt.

\begin{minipage}{0.5\textwidth}
    \begin{figure}[H]
        \centering
        \includegraphics [angle=0, width=0.8\textwidth]{Images/teile/plates/plate_outer_front.png}
        \caption{Frontansicht äußere Platte}
        \label{outerfront}
    \end{figure}
\end{minipage}
\begin{minipage}{0.5\textwidth}
    \begin{figure}[H]
        \centering
        \includegraphics [angle=0, width=0.8\textwidth]{Images/teile/plates/plate_outer_top.png}
        \caption{Obersicht äußere Platte}
    \end{figure}
\end{minipage}

\begin{minipage}{0.4\textwidth}
    \begin{figure}[H]
        \centering
        \includegraphics [angle=0, width=0.8\textwidth]{Images/teile/plates/plate_outer_r.png}
        \caption{ausgedruckte äußere Platte}
        \label{outerprint}
    \end{figure}
\end{minipage}
\begin{minipage}{0.6\textwidth}
    \begin{table}[H]
        \centering
        \begin{tabular}{|c|c|}
            \hline
            Name&äußere Handrückenplatte\\
            \hline
            Datei& $\texttt{controller}\_ \texttt{plate}\_ \texttt{outer}$ \\
            \hline
            LxBxH        & 60mm x 36.1mm x 12.6mm                                  \\
            \hline
            Beschr.& \shortstack{Dieses Element erfüllt die Rolle,\\als Basis für die äußeren zwei\\ Finger zu dienen. Sie ist von}\\
                         & \shortstack{der inneren Platte getrennt, um\\volle Bewegungsfreiheit zu\\ermöglichen.} \\
            \hline
        \end{tabular}
        \caption{Details äußere Platte}
    \end{table}
\end{minipage}

\vspace{3mm}

Auf den großen Platten sind Multiplexer für die entsprechenden Finger, sowie Querverbindungen für die Versorgungsspannung montiert.


