Aufgrund beidseitigem Interesse an Robotik, wurde ursprünglich in Diskussion gebracht, eine bionische Roboterhand 
zu entwerfen, welche über einen Controller ansteuerbar sein soll. Aus diesem Projekt hervorgehen erhoffte man 
sich, Wissen in den Gebieten 3D-Druck, Elektronik, Serverprogrammierung und Kommunikationsprotokollen vertiefen zu können.
Um möglichst viel Modularität in dieses Projekt einzubringen, soll der Controller auch isoliert für 3D Applikationen verwendbar sein.
Da allerdings schnell klar war, dass ein derartiges Projekt den Rahmen einer Diplomarbeit sprengen würde, wurde entschieden, 
nur den Controller-Handschuh zu konstruieren und die mechanische Hand als optionales Ziel bzw. späteres Projekt nach der Diplomarbeit 
zu betrachten. 
\\\\
Dieser Handschuh soll mithilfe von 22 Potentiometern, also verstellbaren Widerständen, alle Freiheitsgrade der menschlichen Hand, insbesondere 
die Rotationswerte der Finger aufnehmen und an einen Computer senden, um die gemessenen Werte anhand eines Modells darzustellen. 