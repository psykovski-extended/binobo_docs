Die Diplomarbeit umfasst einen Controllerhandschuh und eine Website, die von einem Server gehostet wird. 

Es gibt zwei voneinander getrennte Server: Webserver und einen Websocketserver. Der Webserver wurde mithile des Java-Frameworks 
\texttt{Spring Boot} entwickelt. Dieser bieten die Möglichkeit sich als Client zu registrieren und einen Blog und einen Emulator zu nutzen. Der Emulator bietet die Möglichkeit zur echtzeitnahen Emulation der erfassten Rotationswerte auf einem 3D-Modell der menschlichen Hand.\newline
Die Echtzeitemulation wird durch einen Websocketserver möglich, welcher Datenaustausch mit minimalen Latenzen ermöglicht.\\
Der Handschuh besteht aus einem konventionellen Nylonhandschuh, der mit 3D-gedruckten Komponenten beaufschlagt wird, sodass die 
Messelektronik, bestehend aus Potentiometern, Multiplexern und einem ESP32-Mikrocontroller, fixiert werden kann. Die Herausforderung
hierbei bestand hauptsächlich darin, passende Elemente zu konstruieren, um möglichst genaue Messungen durchzuführen. 
\\
